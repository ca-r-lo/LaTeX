\documentclass{article}
\title{Week 9}
\author{Louise Carlo Salomon}
\date{}
\input{longdiv.tex}
\usepackage{polynom}
\usepackage{amsmath}
\begin{document}
    \maketitle
    \pagenumbering{gobble}
    Excercise 1:A\\
        1.$f(x)=x^3-5x^2-7x+4$
            \begin{align*}
                f(1)           & = -(1)^3-5(1)^2-7(1)+4=-7 \\
                f(-2)          & = (-2)^3-5(-2)^2-7(-2)+4=-10 \\
                f(\frac{1}{2}) & = (\frac{1}{2})^3-5(\frac{1}{2})^2-7(\frac{1}{2})+4=-0.0625
            \end{align*}
        2.$g(x)=2x^6+3x^4-x^2+3$
            \begin{align*}
                g(2)&=2(2)^6+3(2)^4-(2)^2+3=175\\
                g(3)&=2(3)^6+3(3)^4-(3)^2+3=1695\\
                g(-1)&=2(-1)^6+3(-1)^4-(-1)^2+3=7
            \end{align*}
        3.$h(x)=2x^3-7x+3$
            \begin{align*}
                h(-3)&=2(-3)^3-7(-3)+3=-30\\
                h(5)&=2(5)^3-7(5)+3=218\\
                h(-10)&=2(-10)^3-7(-10)+3=-1927
            \end{align*}
    \par Excercise 1:B\\
        4.Determine if $x-3$is a factor of $P(x)$ where $P(x)=x^4-3x^3-x+3$.
        \begin{equation}
            \polyhornerscheme[x=3]{x^4-3x^3-x+3}
        \end{equation}
        therefore $x-3$ is a factor of $P(x)$.\\\\
        5.Determine if $x-1$ is a factor of $P(x)$ where $P(x)=x^{25}-4$.
        \begin{align*}
                P(1)=(1)^{25}-4=-3
        \end{align*}
        therefore $x-1$ is not a factor of $P(x)$.\\\\
        6. Find k so that $x-2$ is a factor of $P(x)=x^3-kx^2-4x+20$.
        \begin{align*}
                P(2)&=(2)^3-k(2)^2-4(2)+20\\
                   0&=8-4k-8+20\\
                   0&=20-4k\\
                 -20&=-4k\\
                 \frac{-20}{-4}&=\frac{-4k}{-4}\\
                 5&=k
        \end{align*}
        therefore, $k$ should be $5$ so that $x-2$ will be a factor of $P(x)$
    
    
    Excercise 2. Answer is asked.\\
    1. $P(2)$ in $P(x)=x^4+4x^3-x^2-16x-4$
    \begin{align*}
        P(2)&=(2)^4+4(2)^3-(2)^2-16(2)-4\\
        P(2)&=16+32-4-32-4\\
        P(2)&=8
    \end{align*}
    2. Prove $y-3$ is a factor of $3y^3-7y^2-20$. $y=3$
    \begin{align*}
        0&=3(3)^3-7(3)^2-20
        0&=-2
    \end{align*}
    therefore, $y-3$ is not a factor of $3y^3-7y^2-20$.\\\\
    3. Evaluate $P(4)$ where $P(x)=3y^3-7y^2-20$.
    \begin{align*}
        P(4)&=3(4)^3-7(4)^2-20\\
        P(4)&=60
    \end{align*}
    therefore $P(4)=60$.\\\\
    4. Prove $x-1$ is a factor of $P(x)=$.
    \begin{align*}
        P(1)=(1)^2+2(1)+5
        P(1)=8
    \end{align*}
    using synthetic division,
    \begin{equation}
        \polyhornerscheme[x=1]{x^2+2x+5}
    \end{equation}
    \newpage
    therefore, knowing that,
    \begin{center}
        If $P(a)=0$, then $x-a$ is factor of $P(x)$. Conversely, if $x-a$ is a factor of $P(x)$, then $P(a)=0$.
    \end{center}
    and plugging in our case, we find that,
    \begin{center}
        $P(1)\neq 0$ then $x-1$ is not a factor of $P(x)$. Conversely, $x-1$ is not a factor of $P(x)$, then $P(1)\neq 0$. 
    \end{center}
    5.
    \begin{align*}
        P(x)&=5x^3+3x^2-8\\
        P(4)&=5(4)^3+3(4)^2-8\\
        P(4)&=360
    \end{align*}
    therefore, remainder $R=360$.
\end{document}