\documentclass{article}
\title{}
\author{Louise Carlo Salomon}
\date{}
\usepackage{amsmath}
% \usepackage[document]{ragged2e}
\usepackage{array}
\begin{document}


    \begin{center}
        
    \end{center}





    % \maketitle
    % \pagenumbering{gobble}
    % What I Have Learned: Fill in the Table
    % \begin{center}
    %     \begin{tabular}{ | m{4em} | m{1cm} | m{1cm} | m{1.5cm} |}
    %         \hline 
    %         Given &  $a_{5}$ & $a_{7}$ & $a_{9}$ \\ 
    %         \hline
    %         $a_{1}=3$  $r=3$ & 243 & 2187 & 19683 \\  
    %         \hline
    %         $a_{1}=2$  $r=2$ & 32 & 128 & 512 \\
    %         \hline
    %         $a_{1}=-\frac{1}{2}$  $r=-6$ & -648 & -23328 & -839808 \\ 
    %         \hline 
    %     \end{tabular} \\
    % \end{center}  \par 
    % What Can I Do
    % \begin{equation}
    %     A=P \left( 1+\frac{r}{n} \right)^{nt}
    % \end{equation} \par
    % Where A=\emph{final amount}, P=\emph{initial principal balance}, r=\emph{interest rate}, n=\emph{number of times interest applied per time period}, t=\emph{number of time periods elapsed},    
    % \\ \\ 1.
    % \begin{equation}
    %     r=\frac{5}{100}=0.05, \quad P=12000, \quad n=1, \quad t=9
    % \end{equation}
    % \begin{equation}
    %     A=12000\left(1+\frac{0.05}{1}\right)^{1(9)}
    % \end{equation}
    % \begin{equation}
    %     A=12000\left(1.05\right)^9
    % \end{equation}      
    % \begin{equation}
    %     A=18615.93859
    % \end{equation} 
    % therefore, 18615.93859 is the amount of his money by the year 2028. \\
    % \\ 2.
    % \begin{equation}
    %     P=18615.93859-12000=6615.938592, \quad r=\frac{5}{100}=0.05, \quad n=1, \quad t=5
    % \end{equation}
    % \begin{equation}
    %     A=6615.938592\left(1+\frac{0.05}{1}\right)^{1(5)}
    % \end{equation}
    % \begin{equation}
    %     A=6615.938592\left(1.05\right)^{5}
    % \end{equation}
    % \begin{equation}
    %     A=8443.800443
    % \end{equation}
    % A-P = the interest after 5 years,
    % \begin{equation}
    %     8443.800443-6615.938592=1827.861851
    % \end{equation}
    % therefore, the interest after 5 years is 1827.861851. 
    % \newpage
    % Assessment 1: Halfway
    % \begin{center}
    %     \begin{tabular}{|m{1in}|m{1in}|}
    %         \hline
    %         Given & Geometric Mean \\
    %         \hline
    %         1. 3 and 8 & 4.898979 \\
    %         \hline
    %         2. 100 and 25 & 50 \\
    %         \hline
    %         3. $\frac{1}{2}$ and $\frac{1}{8}$ & $\frac{1}{4}$ \\
    %         \hline
    %         4. 3 and $\frac{1}{3}$ & 1 \\
    %         \hline
    %         5. x and $x^7$ & $x^4$ \\
    %         \hline
    %     \end{tabular}
    % \end{center} 

    % Assessment 2: In Between
    % \begin{equation}
    %     1.) \quad 2,\emph{14},\emph{98},686
    % \end{equation}
    % \begin{equation}
    %     2.) \quad \emph{192},24,\emph{3},\emph{3/8},3/64
    % \end{equation}
    % \begin{equation}
    %     3.) \quad \emph{1/2},\emph{1},\emph{2},4,8
    % \end{equation}
    % \begin{equation}
    %     4.) \quad \emph{1/8},1/4,1/2,\emph{1}
    % \end{equation}
    % \begin{equation}
    %     5.) \quad 81,\emph{27},\emph{9}, \emph{3}, \emph{1}, \frac{1}{3}
    % \end{equation}    
    
    
    
    
    
    
    
    
    
    
    
    
    
    
    
    
    
    
    
    
    
    % 1. The top-bottom method is one in which all plans for any event should always come from the higher-ups or someone with the authority to do so.
    % While the Bottom-Up strategy does not care if you are a member of the royal family or a farmer, everyone who is a member of the community can engage and be active in the future of their town. In brief, citizens have the capacity to modify their community in to something they desire, but local government participation is still required, whereas the top-bottom method only recognizes someone at the top to rule/plan the community.
    % \\ \par
    % 2. The advantages of a top-bottom approach are that people may not have to work up to something and will simply wait for instructions from the higher-ups' plans; however, the disadvantages are that the higher-ups will not be able to know about what citizens experience and feel, as well as what they truly want.
    % The bottom-up approach's merits are that people in the community can speak about their issues and take action to address them; but, its drawback is that if people don't agree on something and create a ruckus about it, it can disrupt the community's peace and order.
    % \\ \par
    % 3. The top-down strategy is the ideal approach for dealing with problems such as hazard or disaster since residents may know how to make the community look nice, but higher-ups have more influence over the situation, such as when, how, or where the disaster may occur. For these types of disasters, a general plan that everyone in the community follows is preferable since the higher ups receive more comprehensive instructions, resulting in a detailed and safe plan for the residents. 


\end{document}

