\documentclass{article}
\title{Bakuna}
\date{2020-10-13}
\author{Louise Carlo Salomon}

\begin{document}
    \pagenumbering{gobble}
    \maketitle
    % \newpage
    % \pagenumbering{arabic}

    Panahon ng pandemiya ay isa sa mga masamang o di kaaya-ayang mangyari sa atin dahil nag sisimbolo itong wala tayong
    laban sa isang sakit na lumalaganap at tanging pag-iingat na lamang ang ating kayang gawin. Ngunit ngayon na meron 
    ng solusyon o sandata laban sa sakit na ito, bakit parang nag-aalangan pa tayo na gamitin ito. 
    \newline \par
    Isa ang bakuna sa makakatulong sa atin upang maprotektahan ang ating katawan sa mga sakit, kaya ng lumabas na ang 
    mga bakuna para sa sakit na nagdulot ng pandemiya sa atin na COVID-19 ay agad na hinikayat ang publiko sa paglapit 
    sa pamahalaan upang magpaturok nito upang malabanan natin itong sakit at makabalik na tayong lahat sa normal. Pero 
    sabay sa pagpa-laganap ng bakuna ay lumaganap rin ang mga impormasyon na kontra sa mga bakunang ito. Sinsabi nilang 
    hindi raw ito mapag-titiwalaan, gawa raw ito ng gobyerno upang kontrolin tayo sa hinaharap, may mga "\emph{nanobots}"
    raw ang mga bakuna ito na dahang-dahang kokontrolin tayo. Ngunit alam nating lahat na hika-hika lamang ito na walang
    kapani-paniwalang pinagmulan na ang layunin lamang ay iligaw tayo sa tamang desisyon na dapat nating gawin. 
    \newline \par
    Pero ano nga ba ang puno't dulo ng pag aalangan nating sa bakuna? Bumalik tayo sa panahon ng Dengvaxia, isang bakuna na
    para sa sakit na Dengue na marami sa ating bansa. Sa paglabas ng Dengvaxia ay agad na nagpagawa ng malawakang bakunahan
    ang gobyerno at kalagitnaan ng bakunahan ay biglang iniba ng Sanofi-ang gumawa sa Dengvaxia, ang deskripsyon ng kanilang 
    bakuna, Nagdulot ito ng kaba at takot sa mga nabakunahan lalo sa na ang kanilang mga kapamilya. Ito ang ugat ng lahat ng
    mga pag-aalangan, walang tiwala sa mga bakuna, galit sa gobyerno, at iba pa na napukaw uli sa paglabas ng bakuna laban sa
    COVID-19.
    \newline \par
    Dengvaxia ay isa lamang sa mga panahon na kailangan na nating tanggapin at magpatuloy sa ating kasalukuyang pangangailangan dahil
    kung hindi, wala itong magandang dulot sa atin at baka maging dahilan kung may mangyari di natin hangad. Wag tayong magpa-uto 
    sa mga maling impormasyon na lumalaganap. Maging masuri at sigurado sa lahat ng ating gagawin. Tanging bakuna lamang ang ating bato
    laban sa sakit na ito na kumitil at kumikitil pa ng mga tao. Protektahan natin ang ating sarili at ang ating kapamilya.  
\end{document}
