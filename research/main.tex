\documentclass{article}

\usepackage{amsmath}
\title{Determine whether memory recall is affected by age}
\author{Louise Carlo Salomon}
\date{}
\begin{document}
     \maketitle
     \pagenumbering{gobble}
     A memory test was conducted to a group of 10 individuals.
     \\\\ Specific Problem: Is there a significant relationship regarding age and memory recall?
     \\\\  Hypothesis: \\ - \hspace{1cm} $H_0$ - There is no significant relationship regarding age and memory recall. 
     \\ -\hspace{1.2cm} $H_1$ - There is a significant relationship regarding age and memory recall.
     \\\\ Level of Significance: $.05$ or $5\%$
     \\\\ Given Data: 
     \begin{center}
         \begin{tabular}{ccc}
             Individual & Age: X & Memory Recall: Y \\
             A & 50 & 83 \\
             B & 54 & 82 \\
             C & 51 & 80 \\
             D & 46 & 80 \\
             E & 49 & 78 \\
             F & 53 & 77 \\
             G & 43 & 76 \\
             H & 47 & 76 \\
             I & 46 & 72 \\
             J & 41 & 71 \\
         \end{tabular}
     \end{center}
     \newpage
     \section*{Pearson r}
    %  \begin{center}
    %      x = sum of $X$, y = sum of $X$, 
    %  \end{center}
    \begin{center}
    \begin{tabular}{cccc}
            Individual & Age: X & Memory Recall: Y  &$X^2$ \\
            A & 50 & 83 & 2500 \\ 
            B & 54 & 82 & 2916 \\ 
            C & 51 & 80 & 2601 \\ 
            D & 46 & 80 & 2116 \\ 
            E & 49 & 78 & 2401 \\ 
            F & 53 & 77 & 2809 \\ 
            G & 43 & 76 & 1849 \\ 
            H & 47 & 76 & 2209 \\ 
            I & 46 & 72 & 2116 \\ 
            J & 41 & 71 & 1681 \\ 
            n=10& $\sum_{i=1}^{n}X_i=480$ & $\sum_{i=1}^{n}Y_i=775$ & $\sum_{i=1}^{n}X^2_i=23198$ \\ 
            & $\overline{X}=48$ & $\overline{Y}=77.5$
    \end{tabular}
    \end{center}
    \begin{center}
        \begin{tabular}{cc}
             $Y^2$ & XY \\
            6889 & 4150\\
            6724 & 4428\\
            6400 & 4080\\
            6400 & 3680\\
            6084 & 3822\\
            5929 & 4081\\
            5776 & 3268\\
            5776 & 3572\\
            5184 & 3312\\
            5041 & 2911\\
            $\sum_{i=1}^{n}Y^2_i=60203$ & $\sum_{i=1}^{n}XY=37304$
        \end{tabular}
    \end{center}
    \par Pearson r, A.
    \begin{equation}
        r=\frac{\sum_{i=1}^{n}XY-n(\overline{X})(\overline{Y})}{\sqrt{(\sum_{i=
        1}^{n}X^2_i-n(\overline{x})^2)
        (\sum_{i=1}^{n}Y_i-n(\overline{y})^2)}}
    \end{equation}

    \begin{align*}
        r&=\frac{37304-10(48)(77.5)}{\sqrt
        {(23198-10(48)^2)(60203-10(77.5)^2)}}\\
        r&=\frac{104}{\sqrt{22199}}\\
        r&=\frac{104}{148.9932884}\\
        r&=0.698018018
    \end{align*}
    \newpage
    Pearson r, B.
    \begin{equation}
        r=\frac{n(\sum_{i=1}^{n}XY)-(\sum_{i=1}^{n}X_i)(\sum_{i=1}^{n}Y_i)}{\sqrt{(n(\sum_{i=1}
        ^{n}X^2_i)(\sum_{i=1}^{n}X_i)^2)(n(\sum_{i=1}^{n}Y^2_i)(\sum_{i=1}^{n}Y_i)^2)}}
    \end{equation}
    \begin{align*}
        r&=\frac{10(37304)-(480)(775)}{\sqrt{(10(23198)-(480)^2)(10(60203)-(775)^2)}}\\
        r&=\frac{1040}{\sqrt{2219900}}\\
        r&=\frac{1040}{1489.932884}\\
        r&=0.698018018
    \end{align*}
    The computed r is 0.0698018018 and the table value r at 0.05 level of significance is 
    \begin{center}
        $df=n-2=10-2=8=0.632$
    \end{center}
    since the computed value of r is greater than the table value of r at 0.05 level of significance,
    \begin{center}
        $0.698018018 > 0.632$
    \end{center}
    therefore, the null hypothesis is rejected, indicating that there is a significant relationship regarding age and memory recall in this study.

    \newpage
    \section*{Spearman, rho}
    \begin{center}
        \begin{tabular}{ccccccc}
            Individual & Age: X & Memory Recall: Y & Rank of X($R_x$)&Rank of Y($R_y$)\\
             A & 50 & 83 &4&1\\ 
             B & 54 & 82 &1&2\\
             C & 51 & 80 &3&3.5\\
             D & 46 & 80 &7.5&3.5\\
             E & 49 & 78 &5&5\\
             F & 53 & 77 &2&6\\
             G & 43 & 76 &9&7.5\\
             H & 47 & 76 &6&7.5\\
             I & 46 & 72 &7.5&9\\
             J & 41 & 71 &10&10\\
             $n=10$
        \end{tabular}
    \end{center}
    \begin{center}
        \begin{tabular}{cc}
            Difference Bet.&\\
            $R_x-R_y(D)$&$D^2$\\
            3&9\\
            -1&1\\
            -.5&.25\\
            4&16\\
            0&0\\
            -4&16\\
            1.5&2.25\\ 
            -1.5&2.25\\
            -1.5&2.25\\
            0&0\\
            &$\sum_{i=1}^{n}D^2=49$
        \end{tabular}
    \end{center}
    \begin{equation}
        \rho=1-\frac{6(\sum_{i=1}^{n}D^2)}{n(n^2-1)}
    \end{equation}
    \begin{align*}
        \rho&=1-\frac{6(49)}{10(10-1)}\\
        \rho&=1-\frac{294}{990}\\
        \rho&=1-0.296969697\\
        \rho&=0.70303030303
    \end{align*}
    The computed rho = 0.945454545 and the table value rho with 0.05 level of significance is
    0.5636,  since the computed rho is greater than the table value rho,
    \begin{equation}
        0.70303030303>.5636
    \end{equation} 
    therefore we conclude that our null hypothesis is rejected which means that there is a significant relationship regarding age and memory recall.
\end{document}

% 6889 +4150+ 6724+ 4428 +6400+ 4080+ 6400+ 3680+ 6048+ 3822+ 5929+ 4081 +5776+ 3268 +5776 +3572 +5184 +3312 +5041
